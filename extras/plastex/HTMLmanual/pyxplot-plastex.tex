% pyxplot.tex
%
% The documentation in this file is part of Pyxplot
% <http://www.pyxplot.org.uk>
%
% Copyright (C) 2006-2012 Dominic Ford <coders@pyxplot.org.uk>
%               2009-2012 Ross Church
%
% $Id: pyxplot.tex 21 2007-02-26 15:14:43Z rpc25 $
%
% Pyxplot is free software; you can redistribute it and/or modify it under the
% terms of the GNU General Public License as published by the Free Software
% Foundation; either version 2 of the License, or (at your option) any later
% version.
%
% You should have received a copy of the GNU General Public License along with
% Pyxplot; if not, write to the Free Software Foundation, Inc., 51 Franklin
% Street, Fifth Floor, Boston, MA  02110-1301, USA

% ----------------------------------------------------------------------------

% LaTeX source for the Pyxplot Users' Guide

\documentclass[a4paper,onecolumn,11pt]{book}
\usepackage[dvips]{graphicx}
%\usepackage{amssymb,amsmath,url,lscape,longtable,fancyvrb,makeidx,wasysym}
\usepackage{afterpage,amssymb,amsmath,bbding,color,longtable,multirow,nicefrac,fancyvrb,makeidx,upgreek,wasysym}
\makeindex
\def\version{0.9.0}
\def\reldate{August 2012}
\include{definitions}

%\newcommand\columncolor[1]{}

\newcommand{\url}[1]{#1}

% Make box and example float environments
%\definecolor{LightGrey}{gray}{0.9}
%\usepackage[bf]{caption}
%\usepackage{float}
%\floatstyle{plain}

% Make box float environment
%\newfloat{boxout2}{thp}{lob}
%\floatname{boxout2}{Box}

\newcommand{\boxout}[3]{
\definecolor{boxoutcol}{rgb}{0.9,0.9,0.82}
\newline
\fcolorbox{black}{boxoutcol}{
{\bf\large #1}
\newline\newline
#3
}
\label{#2}
\newline
}

% Make example float environment
%\newfloat{exampletag}{thp}{loe}
%\floatname{exampletag}{Example}

\newcommand{\example}[3]{
\definecolor{excol}{rgb}{0.82,0.9,0.82}
\newline
\fcolorbox{black}{excol}{
{\bf\large Example: #2}
\newline\newline
#3
}
\label{#1}
\newline
}

\newcommand{\nlnp} {\\\hspace{8mm}}
\newcommand{\nlscf}{\newline\noindent}
\newcommand{\nlfcf}{\newline\hspace{8mm}}

\begin{document}

\begin{titlepage}
\normalsize
\begin{center}
{\Huge \bf Pyxplot Users' Guide}\\
\end{center}
\newline
\begin{center}
{\LARGE \bf A Scientific Scripting Language, \\ Graph Plotting Suite and \\ Vector Graphics Toolkit. \\}
\end{center}
\newline
\begin{center}
{\Large Version \version \\}
\end{center}
\newline
\begin{center}
{\large
Lead Developer: Dominic Ford \\
\vspace{1mm}
Lead Tester: Ross Church \\
\vspace{2mm}
Email: \noindent {\tt coders@pyxplot.org.uk} \\
}
\end{center}
\newline
\begin{center}
{\Large \reldate \\}
\end{center}
\end{titlepage}

\part{Introduction to Pyxplot}
\include{introduction}
\include{installation}
\include{first_steps}
\include{calculations}
\include{data}
\include{programming}
\include{flowctrl}
\part{Plotting and vector graphics}
\setcounter{chapter}{7}
\include{plotting}
\include{terminals}
\include{vector_graphics}
\part{Reference manual}
\setcounter{chapter}{10}
\include{reference}
\include{functions}
\include{types}
\include{constants}
\include{units}
\include{papersizes}
\include{colors}
\include{linestyles}
\include{configuration}
\part{Appendices}
\appendix
\include{other_apps}
\include{gnuplot_diffs}
\include{fit_maths}
\include{changelog}

\printindex

\end{document}
