\documentclass[a4paper,onecolumn,10pt]{article}
\usepackage[dvips]{graphicx}
\usepackage{fancyhdr,url}
\pagestyle{fancy}

\lhead{\it FLATPACK ASTROLABE KIT}
\chead{}
\rhead{\thepage}
\lfoot{}\rfoot{}
\cfoot{\bf\footnotesize\copyright\ 2010 Dominic Ford \& Katie Birkwood. %All Rights Reserved. Permission is granted to copy, distribute and/or modify this document under the terms of the GNU Free Documentation License, Version~1.3 or any later version published by the Free Software Foundation; with no Invariant Sections, with no Front-Cover Texts, and with no Back-Cover Texts.
}

\fancypagestyle{plain}{%
\fancyhf{} % clear all header and footer fields
\fancyfoot[C]{\bf\footnotesize\copyright\ 2010 Dominic Ford \& Katie Birkwood. %All Rights Reserved. Permission is granted to copy, distribute and/or modify this document under the terms of the GNU Free Documentation License, Version~1.3 or any later version published by the Free Software Foundation; with no Invariant Sections, with no Front-Cover Texts, and with no Back-Cover Texts.
}
\renewcommand{\headrulewidth}{0pt}
\renewcommand{\footrulewidth}{0pt}}

%\setlength{\textwidth}{21cm}
%\setlength{\oddsidemargin}{-1in}
\addtolength{\textheight}{-13mm}

%\addtolength{\textwidth}{-7mm}
%\addtolength{\oddsidemargin}{-4cm}

\renewcommand{\thefootnote}{\fnsymbol{footnote}}
\setcounter{footnote}{1}

\title{\vspace{-3cm}Flatpack Astrolabe Kit (Historical Version)}
\author{Dominic Ford\footnote{Cavendish Laboratory, J.J.\ Thomson Ave, Cambridge, CB3 0HE, UK.}~~and Katie Birkwood\footnote{St John's College Library, Cambridge, CB2 1TP, UK.}}
\date{??th February 2010}

\begin{document}
\maketitle
\setcounter{footnote}{3}

This document contains a series of images which should be printed and cut out
to make a simple cardboard version of a medieval astrolabe. The resulting
astrolabe is a simplified and modernised version of that described in Geoffrey
Chaucer's c.\ 1391 {\it Treatise on the Astrolabe}, and has been constructed
following the prescription described in Eisner (1975, 1976a, 1976b). The
diagrams in this document have been produced using
PyXPlot.\footnote{\url{http://www.pyxplot.org.uk}}

\section*{Instructions}

Pages~\pageref{mother_back}, \pageref{mother_front}, \pageref{plate}
and~\pageref{rule} should be printed out onto paper, or more preferably onto
thin card.  Page~\pageref{rete} should be printed onto a sheet of transparent
acetate. The two sides of the Mother (pages~\pageref{mother_back}
and~\pageref{mother_front}) should be glued rigidly back-to-back, perhaps
sandwiching a piece of rigid card. The Plate (page~\pageref{plate}) should be
placed in the space left in the centre of the front of the Mother, with the tab
at the top aligned with the corresponding gap in the degrees scale on the
Mother next to midday, marked with a cross. A medieval astrolabe would have had
many plates prepared for different latitudes; we provide a single plate
prepared for a latitude of $52^\circ$N, suitable for use in northern Europe and
the northern USA. The Rete, printed onto transparent acetate, should be placed
over the Plate.  The Label and the Rule should be placed on either side of the
astrolabe: the Rule, marked out with a declination scale, should be placed over
the Plate and the Rete, whilst the Label should be placed on the back of the
Mother. The whole construction may then be fastened together by placing a
split-pin paper fastener through the centre.

\begin{thebibliography}{9}
\bibitem{chaucer}Chaucer, G., \textit{Treatise on the Astrolabe}, in {\it The Riverside Chaucer}, ed.\ L.D.\ Benson (Boston, 1987)
\bibitem{pap1}Eisner, S., \textit{J.\ Brit. astr. Ass.}, \textbf{86}(1), 18-29 (1975)
\bibitem{pap2}Eisner, S., \textit{J.\ Brit. astr. Ass.}, \textbf{86}(2), 125-132 (1976a)
\bibitem{pap3}Eisner, S., \textit{J.\ Brit. astr. Ass.}, \textbf{86}(3), 219-227 (1976b)
\end{thebibliography}

\newpage
\section*{Mother: Back}
\label{mother_back}
\vspace{-1cm}\centerline{\includegraphics{mother_back.eps}}
\section*{Mother: Front}
\label{mother_front}
\vspace{-1cm}\centerline{\includegraphics{mother_front.eps}}
\section*{The Plate}
\label{plate}
\vspace{1.5cm}\centerline{\includegraphics{plate.eps}}
\section*{The Rete}
\label{rete}
\vspace{1.5cm}\centerline{\includegraphics{rete.eps}}
\section*{The Label and Rule}
\label{rule}
\centerline{\includegraphics{rule.eps}}
\end{document}

